%* pbn.tex *%

\documentclass[11pt,draft]{article}
\usepackage{amssymb}

\begin{document}

\title{Solving Paint-By-Numbers Puzzles with GLPK}

\author{Andrew Makhorin {\tt<mao@gnu.org>}}

\date{August 2011}

\maketitle

\section{Introduction$^1$}

\footnotetext[1]{This section is based on the material from [1].}

A {\it paint-by-numbers} puzzle consists of an $m\times n$ grid of
pixels (the {\it canvas}) together with $m+n$ {\it cluster-size
sequences}, one for each row and column. The goal is to paint the canvas
with a picture that satisfies the following constraints:

1. Each pixel must be blank or white.

2. If a row or column has cluster-size sequence $s_1$, $s_2$, \dots,
$s_k$, then it must contain $k$ clusters of black pixels---the first
with $s_1$ black pixels, the second with $s_2$ black pixels, and so on.

It should be noted that ``first'' means ``leftmost'' for rows and
``topmost'' for columns, and that rows and columns need not begin or end
with black pixels.

\subsubsection*{Example}

\def\arraystretch{.8}

\begin{center}
\begin{tabular}{*{3}{@{$\;\;$}c}c*{10}{@{\ }c}@{}}
 & & && & &1& &1& & & & & \\
 & & && & &1& &1& & & & & \\
 & & &&2&1&1&1&1&1&2&3& & \\
 & & &&3&2&1&2&1&2&3&4&8&9\\
\\
 &3&6&&$\blacksquare$&$\blacksquare$&$\blacksquare$&$\square$&
$\blacksquare$&$\blacksquare$&$\blacksquare$&$\blacksquare$&
$\blacksquare$&$\blacksquare$\\
 &1&4&&$\blacksquare$&$\square$&$\square$&$\square$&$\square$&
$\square$&$\blacksquare$&$\blacksquare$&$\blacksquare$&$\blacksquare$\\
1&1&3&&$\square$&$\square$&$\blacksquare$&$\square$&$\blacksquare$&
$\square$&$\square$&$\blacksquare$&$\blacksquare$&$\blacksquare$\\
 & &2&&$\square$&$\square$&$\square$&$\square$&$\square$&$\square$&
$\square$&$\square$&$\blacksquare$&$\blacksquare$\\
 &3&3&&$\square$&$\square$&$\blacksquare$&$\blacksquare$&$\blacksquare$&
$\square$&$\square$&$\blacksquare$&$\blacksquare$&$\blacksquare$\\
 &1&4&&$\blacksquare$&$\square$&$\square$&$\square$&$\square$&$\square$&
$\blacksquare$&$\blacksquare$&$\blacksquare$&$\blacksquare$\\
 &2&5&&$\blacksquare$&$\blacksquare$&$\square$&$\square$&$\square$&
$\blacksquare$&$\blacksquare$&$\blacksquare$&$\blacksquare$&
$\blacksquare$\\
 &2&5&&$\blacksquare$&$\blacksquare$&$\square$&$\square$&$\square$&
$\blacksquare$&$\blacksquare$&$\blacksquare$&$\blacksquare$&
$\blacksquare$\\
 &1&1&&$\square$&$\square$&$\square$&$\blacksquare$&$\square$&$\square$&
$\square$&$\square$&$\square$&$\blacksquare$\\
 & &3&&$\square$&$\square$&$\blacksquare$&$\blacksquare$&$\blacksquare$&
$\square$&$\square$&$\square$&$\square$&$\square$\\
\end{tabular}
\end{center}

\def\arraystretch{1}

\section{Solving a puzzle}

The Paint-By-Numbers puzzle can be formulated as a 0-1 integer
feasibility problem. The formulation used in GLPK was proposed in [1].

For solving puzzles there are used two components, which both are
coded in the GNU MathProg modeling language [2]: the model section and
the data section. The model section is common for all puzzles and
placed in file \verb|pbn.mod|. This file is included in the GLPK
distribution and can be found in subdirectory \verb|examples/pbn|.

To solve a particular puzzle the user only needs to prepare the data
section, which defines input data to the puzzle. The data section for
the example puzzle from the previous section may look like follows
(here \verb|m| is the number of rows, and \verb|n| is the number of
columns):

\begin{footnotesize}
\begin{verbatim}
data;

param m := 10;

param n := 10;

param row : 1  2  3  :=
      1     3  6  .
      2     1  4  .
      3     1  1  3
      4     2  .  .
      5     3  3  .
      6     1  4  .
      7     2  5  .
      8     2  5  .
      9     1  1  .
      10    3  .  .
;

param col : 1  2  3  4  :=
      1     2  3  .  .
      2     1  2  .  .
      3     1  1  1  1
      4     1  2  .  .
      5     1  1  1  1
      6     1  2  .  .
      7     2  3  .  .
      8     3  4  .  .
      9     8  .  .  .
      10    9  .  .  .
;

end;
\end{verbatim}
\end{footnotesize}

\newpage

Let the data section for a puzzle be placed in file \verb|foo.dat|.
Then to solve the puzzle the user should enter the following command:

\begin{verbatim}
   glpsol --minisat -m pbn.mod -d foo.dat
\end{verbatim}

\noindent
This command invokes \verb|glpsol|, the GLPK LP/MIP stand-alone solver,
which reads the model section from file \verb|pbn.mod|, the data section
from file \verb|foo.dat|, translates them to an internal representation,
and solves the resulting 0-1 integer feasibility problem. The option
\verb|--minisat| tells \verb|glpsol| to translate the feasibility
problem to a CNF satisfiability problem and then use the MiniSat solver
[3] to solve it.

If a solution to the puzzle has been found, that is indicated by the
message \verb|SATISFIABLE|, \verb|glpsol| prints the solution to the
standard output (terminal), writes it to file \verb|solution.ps| in the
PostScript format, and also writes it to file \verb|solution.dat| in the
form of MathProg data section, which can be used later to check for
multiple solutions, if necessary (for details see the next section).
The message \verb|UNSATISFIABLE| means that the puzzle has no solution.

Usually the time taken to solve a puzzle of moderate size (up to 50 rows
and columns) varies from several seconds to several minutes. However,
hard or large puzzles may require much more time.

Data sections for some example puzzles included in the GLPK distribution
can be found in subdirectory \verb|examples/pbn|.

\section{Checking for multiple solutions}

Sometimes the user may be interested to know if the puzzle has exactly
one (unique) solution or it has multiple solutions. To check that the
user should solve the puzzle as explained above in the previous section
and then enter the following command:

\begin{verbatim}
   glpsol --minisat -m pbn.mod -d foo.dat -d solution.dat
\end{verbatim}

\noindent
In this case \verb|glpsol| reads an additional data section from file
\verb|solution.dat|, which contains the previously found solution,
activates an additional constraint in the model section to forbid
finding the solution specified in the additional data section, and
attempts to find another solution. The message \verb|UNSATISFIABLE|
reported by \verb|glpsol| will mean that the puzzle has a unique
solution, while the message \verb|SATISFIABLE| will mean that the puzzle
has at least two different solutions.

\newpage

\section{Solution times}

The table on the next page shows solution times on a sample set of
the paint-by-numbers puzzles from the \verb|<webpbn.com>| website.
This sample set was used in the survey [4] to compare efficiency of
existing PBN solvers.

The authors of some puzzles from the sample set have given permission
for their puzzles to be freely redistributed as long as the original
attribution and copyright statement are retained. In the table these
puzzles are marked by an asterisk (*). The files containing the
MathProg data sections for these puzzles are included in the GLPK
distribution and can be found in subdirectory \verb|examples/pbn|.

All runs were performed on Intel Pentium 4 (CPU 3GHz, 2GB of RAM).
The C compiler used was GCC 3.4.4 with default optimization options.

The column `Sol.Time' shows the time, in seconds, taken by the
\verb|glpsol| solver to find a solution to corresponding puzzle. The
column `Chk.Time' shows the time, in seconds, taken by \verb|glpsol| to
check for multiple solutions, i.e. either to prove that the puzzle has
a unique solution or find another solution to the puzzle. Both these
times do not include the time used to translate the MathProg model and
data sections into an internal MIP representation, but include the time
used to translate the 0-1 feasibility problem to a CNF satisfiability
problem.

\begin{thebibliography}{10}

\bibitem{1}
Robert A. Bosch, ``Painting by Numbers'', 2000.\\
\verb|<http://www.oberlin.edu/~math/faculty/bosch/pbn-page.html>|.

\bibitem{2}
GLPK: Modeling Language GNU MathProg. Language Reference. (This
document is included in the GLPK distribution and can be found in
subdirectory \verb|doc|.)

\bibitem{3}
Niklas E\'en, Niklas S\"orensson, ``An Extensible SAT-solver'',
Chalmers University of Technology, Sweden. \verb|<http://minisat.se/>|.

\bibitem{4}
Jan Wolter, ``Survey of Paint-by-Number Puzzle Solvers''.\\
\verb|<http://webpbn.com/survey/>|.

\end{thebibliography}

\newpage

\begin{table}
\caption{Solution times on the sample set of puzzles from [4]}
\begin{center}
\begin{tabular}{@{}lllcrr@{}}
\hline
\multicolumn{2}{c}{Puzzle}&Size&Notes&Sol.Time, s&Chk.Time, s\\
\hline
\#1&Dancer*    &$10\times 5$&L&$<1$&$<1$\\
\#6&Cat*       &$20\times 20$&L&$<1$&$<1$\\
\#21&Skid*     &$25\times 14$&L, B&$<1$&$<1$\\
\#27&Bucks*    &$23\times 27$&B&$<1$&$<1$\\
\#23&Edge*     &$11\times 10$&&$<1$&$<1$\\
\#2413&Smoke   &$20\times 20$&&$<1$&$<1$\\
\#16&Knot*     &$34\times 34$&L&1&1\\
\#529&Swing*   &$45\times 45$&L&1&1\\
\#65&Mum*      &$40\times 34$&&1&1\\
\#7604&DiCap   &$55\times 55$&&10&10\\
\#1694&Tragic  &$50\times 45$&&3&3\\
\#1611&Merka   &$60\times 55$&B&4&4\\
\#436&Petro*   &$35\times 40$&&1&1\\
\#4645&M\&M    &$70\times 50$&B&5&6\\
\#3541&Signed  &$50\times 60$&&7&7\\
\#803&Light*   &$45\times 50$&B&1&1\\
\#6574&Forever*&$25\times 25$&&1&1\\
\#2040&Hot     &$60\times 55$&&6&6\\
\#6739&Karate  &$40\times 40$&M&2&2\\
\#8098&9-Dom*  &$19\times 19$&&1&2\\
\#2556&Flag    &$45\times 65$&M, B&2&2\\
\#2712&Lion    &$47\times 47$&M&11&12\\
\#10088&Marley &$63\times 52$&M&135&226\\
\#9892&Nature  &$40\times 50$&M&850&1053\\
\hline
\end{tabular}

\begin{tabular}{@{}lp{102mm}@{}}
*&Puzzle designer has given permission to redistribute the puzzle.\\
L&Puzzle is line solvable. That is, it can be solved one line at a
time.\\
B&Puzzle contains blank rows or columns.\\
M&Puzzle has multiple solutions.\\
\end{tabular}
\end{center}
\end{table}

\end{document}
